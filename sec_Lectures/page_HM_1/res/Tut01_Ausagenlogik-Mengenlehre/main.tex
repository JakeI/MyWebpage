\documentclass{beamer}
 
\usepackage[utf8]{inputenc}
 
\title{HM1 - Tut 01 - Aussagenlogik und Mengenlehre}
\author{Jochen Illerhaus}
\institute{KIT}
\date{24.10.17}
 
\begin{document}
 
\frame{\titlepage}
 
\begin{frame}
    \frametitle{Allgemeines}
    \begin{itemize}
        \item Siehe Webseite \url{https://goo.gl/isqjdv}
        \item URL steht sonnst nirgends.
        \item Nächste Wochen (31.10.17) keine Tutorium.
    \end{itemize}
\end{frame}

\begin{frame}
    \frametitle{Fragen}
    
    \huge Gibt es Fragen?
    
\end{frame}

\begin{frame}
    \frametitle{Aussagenlogik}
     \begin{itemize}
        \item Erster Kontakt mit "Echter Mathematik"
        \item Hier ein Extrembeispiel:
            \begin{itemize}
                \item \textbf{Def} Schul-Mathematik $S$: Menge der in der Schule Benutzten Methoden.
                \item \textbf{Def} Universitäts-Mathematik $U$: Menge der an Universitäten Benutzten Methoden.
                \item \textbf{Thm} $\forall S, U: S \cap U \approx \emptyset$
                \item \textbf{Proof} omitted
            \end{itemize}
        \item Ist "mathematisch" für: "Schule und Uni unterscheiden sich"
        \item Das ist aber trotzdem sinfoll!!!
    \end{itemize}
\end{frame}

\begin{frame}
    \frametitle{Aussagenlogik}
    \begin{itemize}
        \item Darstellung von Wahr/Falsch ($w$/$f$) Fragen als Kleinbuchstaben:
        \item 
            \begin{center}
                \begin{tabular}{c c | c | c | c}
                    $p$ & $q$ & $p \wedge q$ & $p \vee q$ & $\neg p$ \\
                    \hline
                    $f$ & $f$ & $f$ & $f$ & $w$  \\
                    $f$ & $w$ & $f$ & $w$ & $w$  \\
                    $w$ & $f$ & $f$ & $w$ & $f$  \\
                    $w$ & $w$ & $w$ & $w$ & $f$  \\
                \end{tabular}
            \end{center}
        \item Siehe Außerdem $p \Rightarrow q$, $p \Leftrightarrow q$ und De Morgansche Gesetze.
    \end{itemize}
\end{frame}

\begin{frame}
    \frametitle{Aufgabe 1 und 2}
    Negieren Sie folgende Aussagen:\\
    \textbf{1c)} Wenn morgen schönes Wetter ist, gehen alle Studierenden in den Schlossgarten.\\
    \textbf{1d)} Es gibt einen Menschen, dem Mathematik keinen Spaß macht.\\
    \vspace{1cm}
    \textbf{2)} Betrachten Sie die beiden Aussagen $K$: ”Peter hat kein Kind“ und $T$: ”Peter hat keine Tochter“. Was lässt sich über die Aussagen $K \Leftarrow T$ bzw. $T \Leftarrow K$ sagen?
\end{frame}

\begin{frame}
    \frametitle{Mengenlehre}
    \begin{itemize}
        \item Erweiterung der Aussagenlogik für viele Aussagen.
        \item Später Allgemeiner Umgang mit (unendlich) vielen mathematischen Objekten.
    \end{itemize}
    \begin{figure}
    	\centering
    	\def\svgwidth{\columnwidth}
        \input{Mengen.pdf_tex}
    	\label{fig:kkk}
    \end{figure}
\end{frame}

\begin{frame}
    \frametitle{Aufgabe 3}
    Fur jedes $j \in \mathbb{N}$ sei die Menge
    \[
    S_j := \{ x : x \text{ studiert in Karlsruhe und ist im $j$-ten Hochschulsemester} \}
    \]
    gegeben. Weiter seien $E$, $P$ bzw. $G$ die Mengen der Elektrotechnik-, Physik- bzw. Geodäsie Studierenden
    in Karlsruhe. Drucken Sie folgende Mengen mittels $S_j$
    , $E$, $P$ und $G$ aus:\\
    \textbf{a)} Die Menge all derer, die in Karlsruhe im ersten Hochschulsemester sind und Physik
    studieren.\\
    \textbf{b)} Die Menge aller Karlsruher Studierenden, die im ersten oder dritten Hochschulsemester
    sind, aber nicht Elektrotechnik studieren.\\
    \textbf{c)} Die Menge aller Studierenden in Karlsruhe.\\
\end{frame}
 
\begin{frame}
    \frametitle{Fragen}
    
    \huge Gibt es Fragen?
    
\end{frame}
 
\end{document}